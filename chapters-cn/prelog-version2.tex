\thispagestyle{empty}
\chapter*{第二版序}
《视觉 SLAM 十四讲:从理论到实践》出版已经两年多。两年来,这本书经历了 13 次重印,在 GitHub 上拥有 2.5k 星星,也在业界引起了广泛的关注和讨论。大多数读者评价是正面的,当然,书中也有些地方不够令人满意。例如,这本书作为初学者向,入门则入门矣,有些应该深入的地方讲得不够深入;书中的数学符号不够统一,有些地方容易令读者产生误解;工程实践章节内容不够丰富,介绍较浅,等等。实际上,第1版在 2016 年中期开始创作,所有文字、图片和代码都是从0开始准备,再加上我当时在读博士,也是第一次写这么厚的书,错漏在所难免。2018 年,我在慕尼黑工大给学生讲 SLAM 课程,期间又积累了一些材料,所以本书从内容上更丰富,更合理。在第1版的基础上做了如下改动:

\begin{enumerate}
\item 更多的实例。我增加了一些实验代码来介绍算法的原理。在第1版中,多数实践代码调用了各种库中的内置函数,我现在认为更深入地介绍底层计算会更好。所以本书中的许多代码,在调用库函数之外,还提供了底层的实现。
\item 更深入的内容,特别是从第7讲至第12讲的部分,同时删除了一些泛泛而谈的边角料(比如 GTSAM 相关内容➀)。对第1版大部分数学公式进行了审查,重写了那些容易引起误解的地方。
\item 更完善的工程项目。我将第1版的第9讲移至第13讲。于是,我们可以在介绍了所有必要知识之后,向大家展现一个完整的 SLAM 系统如何工作。相比于第1版,我在本书的项目中将追求以精简的代码实现完整的功能,你会得到一个由几百行代码实现的,有完整前后端的 SLAM 系统。
\item 更通俗、简洁的表达。我觉得这是一本好书的标准,特别是介绍一些看起来高深莫测的数学知识时。我重新制作了部分插图,使它们在黑白印刷之后看起来更清楚。
\item 当然,每讲前的简笔画我是不会改的!
\end{enumerate}
总之,我尽量做到深入浅出,也希望本书能够给你更加舒适的阅读体验。

\clearpage